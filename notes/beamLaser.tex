\documentclass[aps, superscriptaddress, groupedaddress, preprint]{revtex4}

\usepackage{amssymb}
\usepackage{amsmath}
\usepackage{float}
\usepackage{graphicx}

\begin{document}

\tableofcontents

\section{Model}

Atoms move ballistically.  Emitted from an oven at a temperature
$T$ with mean velocity $\bar{v}$.  Collimated to an angle
$\alpha$ in direction transverse to beam.  With flux $\dot N$,
density $\rho$.

We model them with particles of weight $w$, meaning that a
particle in the simulation corresponds to $w$ real particles.
There is an assumption here about how the dipoles due to the
different atoms add up: Coherent addition.  Good above threshold
but needs to be verified by scaling $w$ down.

Field: single mode with resonance frequency $\omega_c$ and decay
rate $\gamma$.  Mode function,
$\vec{\epsilon}(\vec{x})\varphi(\vec{x})$, with $\vec{\epsilon}$
the polarization unit vector and $\varphi$ normalized to unity in
$L_2$,
\begin{equation}
\int d^3\mathbf{x} \left|\varphi(\mathbf{x})\right|^2 = 1\;.
\end{equation}

Dipole interaction between atoms and cavity:

\begin{equation}
\hat H_I=\mathbf{d}\cdot\mathbf{E}(\mathbf{x})
\end{equation}
Inserting $\mathbf{E}=\epsilon\varphi\hat a +
\epsilon^*\varphi^* \hat a^\dagger$ we find
\begin{equation}
\hat H_I =
\frac{\hbar g_{ F^\prime, m_F^\prime;F, m_F}(\vec{x})}{2}
\left|F^\prime,m_F^\prime\right>\left<F, m_F\right|\hat a
+
\frac{\hbar g_{F, m_F;F^\prime, m_F^\prime}(\vec{x})}{2}
\left|F, m_F\right>\left<F^\prime,m_F^\prime\right|\hat a^\dagger
\end{equation}
where
\begin{equation}
g_{ F^\prime, m_F^\prime;F, m_F}=
\left<F^\prime,m_F^\prime\right|
\vec{\epsilon}\cdot\vec{d}(\vec{x})\varphi(\vec{x})
\left|F,m_F\right>
\end{equation}

\paragraph{Equation of motion for the field} We find the
interaction piece of the equation of motion for the field using
the Heisenberg equation of motion
\begin{equation}
i\frac{d\hat a}{dt}=
\frac{1}{2}\sum_{m_F,m_F^\prime}
g_{F, m_F;F^\prime, m_F^\prime}
\left<\psi|F^\prime, m_F^\prime\right>
\left<F, m_F|\psi\right>
\end{equation}
We work in the frame rotating at the frequency of the cavity
field.  In addition to the interaction piece there are damping
and fluctuations.
\begin{equation}
i\frac{d\hat a}{dt}=
\frac{1}{2}\sum_{m_F,m_F^\prime}
g_{F, m_F;F^\prime, m_F^\prime}
\left<\psi|F^\prime, m_F^\prime\right>
\left<F, m_F|\psi\right>
-\frac{\kappa}{2}\hat a + \hat F_a\;.
\end{equation}
The noise term has zero mean and has correlations
%TODO: need to check factors of two here
\begin{eqnarray}
\left<\hat F_a\hat F_a^\dagger\right>&=&\delta(t-t^\prime)\kappa\\
\left<\hat F_a^\dagger\hat F_a\right>&=&0\;.
\end{eqnarray}
Now we need to go over to semiclassical equations of motion by
introducing c-number stochastic variables.  Need to find the
diffusion matrix for the corresponding semiclassical noise terms.
And then we need to do an operator splitting.  The free damped
equations can be integrated analytically.  The interaction piece
has to be dealt with using a numerical integrator, e.g. RK4.



\section{Simulation algorithm}

\section{Implementation notes}


\subsection{Algorithm for computation of atom field interaction}

A key part of the update sequence is the evolution of the coupled
atom-cavity system.  During this step, the atoms are at rest
because their motion is dealt with during the push.  The
mode-function is thus constant at each atoms' location.  The
polarization changes due the atomic internal state changing and
the field amplitude changes as well.

\subsubsection{Inputs}

\begin{itemize}
\item Field amplitude
\item Atomic internal states
\item Mode function at each atoms location
\end{itemize}


\subsubsection{Outputs}

\begin{itemize}
\item Field amplitude at time $t+\Delta t$
\item Atomic internal state at time $t+\Delta t$
\end{itemize}


\subsubsection{Update algorithm}

RK4 with the following RHS:

\begin{itemize}
\item Scatter field amplitude from Rank 0
\item {\it polarization} = 0
\item For each atom:
\begin{itemize}
  \item Load internal state $|\psi\rangle$
  \item Compute $|\psi^\prime\rangle =
                 \varphi\vec{\epsilon}\cdot\vec{d}|\psi\rangle$
  \item {\it polarization} += $\langle\psi|\psi^\prime\rangle$
  \item Store $|\psi^\prime\rangle$
\end{itemize}
\item {\it polarization} is the rhs for the field amplitude and
the $|\psi^\prime\rangle$ are the rhs for the atomic internal
state.
\end{itemize}

\paragraph{Comment on units} We need to figure out how to
parametrize $\vec{d}$.  The dipole matrix element should be
pulled out and combined with the electric field.  We are thus
left with evaluating the matrix vector application for the
operator $\vec{L}$.

\paragraph{Comment on matrix structure of $\vec{L}$} In the
canonical $|F,m_F\rangle$ basis the operator $\vec{L}$ is
tridiagonal.  For computational efficiency we should exploit this
structure.  It may be beneficial to use $J_z$, $J_+$, and $J_-$
instead of $J_z$, $J_x$, and $J_y$ because that way we can
operate on one diagonal at a time.  Perhaps this transformation
should be carried all the way through to the representation of
the mode function.  That way we don't have to convert between
$E_x,E_y$ and $E_+, E_-$ all the time.


\subsection{TODO list}

\paragraph{Get rid of Ring Buffer for particles {\bf [done]}} The
idea of the ring buffer data structure was to make it cheap to
add and remove particles from the simulation.  However, this
unduly penalizes other parts of the simulation that are
relatively more expensive.  In particular, it prevents
vectorization of many important loops.  In addition it makes much
of the code more complicated and leads to ambiguities where it is
not immediately apparent whether an array is a ring buffer or an
array.  Also, this is not needed for many simulations that don't
insert / remove particles frequently.

Removal and insertion of particles are affected by this change.
Particles should be inserted at the front of arrays (this
involves a \verb~memmove~; the only additional cost of using
arrays rather than a ring buffer) and particles should be deleted
from the end using a BSP similar to what's done in the ring
buffer implementation at the front.

\paragraph{Need to scatter field amplitudes from rank 0}
Strictly speaking this is not needed as long as the field evolves
identically on each node (e.g. in the absence of noise in the
field equations).  But as soon as we add noise the field
amplitudes evolve differently on each node and only one of them
is the true amplitude.  We can still evolve the fields using the
same equations on each processor (as long as that's not too much
work) to simplify the control flow.  I.e. no need to add 
\verb~if (!rank) { ... }~ around the field updates.  The
amplitudes on ranks other than zero simply get overwritten.

\paragraph{Parallel initialization of random number generation
{\bf [needs testing]}}
The \verb~sprng~ library needs to be compiled and included
differently depending on whether it is used with MPI.


\end{document}

% vim: nocindent tw=65 spell

